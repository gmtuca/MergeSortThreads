\documentclass{article}
\usepackage{graphicx}
\usepackage{listings}
\usepackage{color}

\definecolor{dkgreen}{rgb}{0,0.6,0}
\definecolor{gray}{rgb}{0.5,0.5,0.5}
\definecolor{mauve}{rgb}{0.58,0,0.82}
\definecolor{orange}{rgb}{1.0,0.5,0}

\lstset{frame=tb,
  language=Java,
  aboveskip=3mm,
  belowskip=3mm,
  showstringspaces=false,
  columns=flexible,
  basicstyle={\small\ttfamily},
  numbers=none,
  numberstyle=\color{orange},
  keywordstyle=\color{blue},
  commentstyle=\color{dkgreen},
  stringstyle=\color{mauve},
  breaklines=true,
  breakatwhitespace=true,
  tabsize=3
}

\graphicspath{ {out/} }

\title{Analysis of Merge Sort with Threads}
\date{12-04-2016}
\author{Arthur Ceccotti - 8544173}

\begin{document}
  \maketitle
  I have written a version of Merge Sort of a vector of random integer contents in Java. The correctness of the program was guaranteed by running it against a sequence of JUnit tests developed in advance (attached on the bottom of this report).

  Upon completion of the application, I evaluated its performance by plotting it with different parameters (vector size and number of threads). Intermittent performance drops/spikes were reduced by running every instance 10 times, allowing the gathering of average values and standard deviation. The large amount of data produced from all the runs was analysed and plotted by my Python script available in this report. 

  With the increase of threads, one would expect linear speed up of performance, as load is equal for every thread and there are enough hardware cores. This means, with the same vector size, 10 threads should run twice faster than 5 threads. This O(\textit{n})speedup is certainly not true for the sorting of small vectors, because thread initialisation and joins would be a large overhead (which may take longer than the actual sorting!).

  %%%%%%%%%%%%%%%%%%%%%%%%%%

  At first I thought the increase of threads, given constant vector size, would result on a linear increase of performance. Because each thread has equal load (with at most one more operation), vector addition with 4 threads would ideally run twice faster than that with 2 threads of same size. Thus my expected speedup was of \textit{n} times compared to a fully sequential execution, where n is the number of active threads.



  Figure 1 describes the performance decrease given increase of number of threads for a small problem size of 10,000 elements on the machine \textit{mcore48.cs.man.ac.uk}  \footnote{\label{machinespecs} \\
  Dell/AMD quad 12-core (48 cores total), each with specs:\\
  model name: AMD Opteron(tm) Processor 6174 \\
  clock speed: 2.2 GHZ \\
  cache size: 512 KB \\}. I evaluate this to occur mostly due to the thread initialisation cost. As we can see, the program performs much better with a single thread for this problem size, which means in this case it is quicker to do the whole vector addition sequentially, rather than spending a long time spawning threads which will die after doing few operations.
\newpage
\begin{figure}[t]
\centerline{%
\includegraphics[width=0.56\textwidth]{10k_lines}%
\includegraphics[width=0.56\textwidth]{10k_errorbars}%
}%
\caption{Parallel vector addition of size 10,000. Average performance (left) and standard deviation error range (right) from 20 runs}
\end{figure}
  
  For larger problem sizes (\textgreater1,000,000), thread initialization of a few milliseconds starts becoming insignificant compared to the lifetime of each thread.
  The plot in figure 2 describes the run of my program for the vector size of 10,000,000 with variable threads, where such performance increase is clearly visible.

  This now shows how advantageous multithreaded programming can be for large input problems when few data dependencies are present.

\begin{figure}[t]
\centerline{%
\includegraphics[width=0.56\textwidth]{10M_lines}%
\includegraphics[width=0.56\textwidth]{10M_errorbars}%
}%
\caption{Parallel vector addition of size 10,000,000. Average performance (left) and standard deviation error range (right) from 20 runs}
\end{figure}

\clearpage

\newpage
\newpage\newpage\newpage\newpage
  \lstinputlisting{src/VectorAdder.java}
  \lstinputlisting{src/VectorAdderThread.java}
  \lstinputlisting{src/VectorAdderTest.java}
\end{document}